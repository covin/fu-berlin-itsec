\documentclass[a4paper,11pt]{article}

\topmargin -2cm
%\topskip0cm
%\footskip0cm
%\headsep0cm
\parindent0cm
\oddsidemargin -1cm
\evensidemargin -1cm
\headheight 2cm
\textheight 24cm
\textwidth 18cm

\author{Alexander Münn}
\title{Exercise}

\usepackage{ucs}
\usepackage[utf8x]{inputenc}
\usepackage{ngerman}
\usepackage[ngerman]{babel}
\usepackage[ngerman]{translator}
\usepackage{color}
\usepackage{url}

\usepackage{graphicx}
\usepackage{algorithmic}
\usepackage{hyperref}

\usepackage{textcomp}

\usepackage{makeidx}
\usepackage{amsmath}
\usepackage{amsfonts}
\usepackage{amssymb,euscript}
\usepackage{dsfont}
\usepackage{fancyhdr}
\usepackage{listings}
\usepackage{enumerate}

\pagestyle{empty}

\lstset{numbers=left,
        basicstyle=\footnotesize,
        numberstyle=\footnotesize,
        stepnumber=5,
        numbersep=5pt,
        frame=tb,
        framexleftmargin=5mm,
        xleftmargin=5mm
}

\pagestyle{fancy}
\lhead{Rechnersicherheit}
\chead{Übung \nr\\\today}
\rhead{Alexander Münn}


\newcommand{\nr}{04}

\begin{document}
\section{Chinese wall -- Bell-LaPadula}
\section{Assembler}
siehe: https://github.com/covin/fu-berlin-itsec/tree/master/e04

\section{Covered Channels}
Für das Encodieren bzw. Leaken von Informationen habe ich mich für Variante 5
aus \textit{A Note on the Confinement Problem. Lampson, ACM 1973} entschieden:
\begin{algorithm}
\caption{Covered Channel -- send logic}
\begin{algorithmic}
\Function{send\_bit}{$b$}
    \If {$b$}
         \State create\_file(bit)
    \EndIf
    \State create\_file(sent)
    \State wait\_for\_file(received)
    \If {$b$}
         \State delete\_file(bit)
    \EndIf
    \State delete\_file(sent)
    \State wait\_for\_deleted(received)
    \State \Return
\EndFunction
\end{algorithmic}
\end{algorithm}

\begin{algorithm}
\caption{Covered Channel -- receive logic}
\begin{algorithmic}
\Function{receive\_bit}{}
    \State wait\_for\_file(sent)
    \If {file\_exist(bit)}
         \State $b \gets 0xFF$
    \Else
         \State $b \gets 0x00$
    \EndIf
    \State create\_file(received)
    \State wait\_for\_deleted(sent)
    \State delete\_file(received)
    \State \Return $b$
\EndFunction
\end{algorithmic}
\end{algorithm}

\lstset{caption={Huffmann table (Generated on http://huffman.ooz.ie/)}}
\begin{lstlisting}
O     = 010
SPACE = 110
N     = 0011
A     = 0111
I     = 1000
S     = 1001
E     = 1011
T     = 1110
R     = 1111
D     = 00001
L     = 00100
B     = 00101
H     = 01100
P     = 01101
F     = 10100
M     = 10101
C     = 000001
W     = 000100
,     = 000101
G     = 000110
U     = 000111
Y     = 0000000
.     = 0000001
\end{lstlisting}


Die Huffmann-Tabelle liegt bei dem Versender als Lookup-Tabelle vor, wobei wir
natürlich auf die Byte-Darstellung des Encodings angewiesen sind. Darum ist es
ebenfalls wichtig, die Anzahl der benötigtien Bits für ein Character-Encoding zu
speichern. Unter der Prämisse das acht Bit für das längste Encoding ausreichen,
benötigen wir für die Übertragung einer Tabelle mit $n$ Elementen $3 \dot n$
Byte. Da wir pro Eintrag eine feste Segmentgröße haben, muss die $n$ initial
übertragen werden, womit dann fest steht, wieviele Bytes bis zum vollständigen
Empfang gelesen werden müssen.

\begin{algorithm}
\caption{Send the message}
\begin{algorithmic}
\Function{send}{c, bits=length(c)}
    \ForAll{bits $b \in c_{0 \to bits}$}
        \State send\_bit($b$)
    \EndFor

\EndFunction

\State $M =$ "The confinement problem, as\ldots"
\Comment build Huffmann encoding table from $M$
\State $dict = $ Map($char \gets (encoding,nbits)$)

\State $l \gets $ dict.length
\State send($l$)
\Comment send Huffmann encoding table
\ForAll{entries $e \in dict$}
    \State send($e.char$)
    \State send($e.encoding$)
    \State send($e.nbits$)
\EndFor
\Comment finally send message
\ForAll {characters $c \in M$}
    \State $e \gets dict$.find($c$)
    \State send($e.encoding$, $e.nbits$)
\EndFor
\end{algorithmic}
\end{algorithm}

\begin{algorithm}
\caption{Receive the message}
\begin{algorithmic}

\Function{receive}{bits=8}
    \State $b \gets 0x00$
    \For{$i = 0 \to bits$}
        \State $b \gets b \| $(receive\_bit() $\& (0x01 << i)$)
    \EndFor
    \State \Return $b$
\EndFunction

\Function{receive\_encoded}{$dict$}
    \State $in \gets 0x00$
    \State $i \gets 0$
    \Repeat
        \State $in \gets in \| $(receive\_bit() $\& (0x01 << i++)$)
    \Until{$(in,i-1) \in dict$}
    \State \Return $dict$.find($in$)
\EndFunction


\State $M = \{\}$
\State $dict = $ Map()

\Comment receive Huffmann encoding table
\State $l \gets $ receive()
\For{$i = 0 \to l$}
    \State $char \gets$ receive()
    \State $enc \gets$ receive()
    \State $bits \gets$ receive()
    \State $dict$.put($(enc,bits) \gets char$)
\EndFor
\Comment receive Huffmann encoded message
\For {$c \gets$ receive\_encoded($dict$)}
    \State $M$.append($c$)
\EndFor
\end{algorithmic}
\end{algorithm}

\end{document}
