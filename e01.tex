\include{tex/00-header-ex}
\usepackage{ucs}
\usepackage[utf8x]{inputenc}
\usepackage{ngerman}
\usepackage[ngerman]{babel}
\usepackage[ngerman]{translator}
\usepackage{color}
\usepackage{url}

\usepackage{graphicx}
\usepackage{algorithmic}
\usepackage{hyperref}

\usepackage{textcomp}

\usepackage{makeidx}
\usepackage{amsmath}
\usepackage{amsfonts}
\usepackage{amssymb,euscript}
\usepackage{dsfont}
\usepackage{fancyhdr}
\usepackage{listings}
\usepackage{enumerate}

\pagestyle{empty}

\lstset{numbers=left,
        basicstyle=\footnotesize,
        numberstyle=\footnotesize,
        stepnumber=5,
        numbersep=5pt,
        frame=tb,
        framexleftmargin=5mm,
        xleftmargin=5mm
}

\pagestyle{fancy}
\lhead{Rechnersicherheit}
\chead{Übung \nr\\\today}
\rhead{Alexander Münn}


\newcommand{\nr}{1}
\setcounter{section}{0}

\begin{document}

\section{State transition}

\subsection*{ACM transition on actions 1-5}

\begin{table}[h]
\caption{Initial access control matrix}
\begin{center}
\begin{tabular}{l|c|c|c|c|c|c|c|}
       & $O_1$ & $O_2$ & $O_3$ &       & $U_1$ & $U_2$ & $U_3$ \\ \hline
 $U_1$ &  r,w  &  r,w  &       &       &   -   &       &       \\ \hline
 $U_2$ &  r,w  &  r,w  &       &       &       &   -   &       \\ \hline
 $U_3$ &  r,w  &  r,w  &  r,w  &       &       &       &   -   \\ \hline
\end{tabular}
\end{center}
\end{table}
\begin{table}[h]
\caption{Access control matrix after 1 and 2}
\begin{center}
\begin{tabular}{l|c|c|c|c|c|c|c|}
       & $O_1$ & $O_2$ & $O_3$ &       & $U_1$ & $U_2$ & $U_3$ \\ \hline
 $U_1$ &  r    &  r,w  &  (r)  &       &   -   &       &       \\ \hline
 $U_2$ &  r,w  &  r,w  &  (r)  &       &       &   -   &       \\ \hline
 $U_3$ &  r,w  &  r,w  &  r,w  &       &       &       &   -   \\ \hline
\end{tabular}
\end{center}
\end{table}
\begin{table}[h]
\caption{Access control matrix after 3 and 4}
\begin{center}
\begin{tabular}{l|c|c|c|c|c|c|c|}
       & $O_1$ & $O_2$ & $O_3$ & $O_4$ & $U_1$ & $U_2$ & $U_3$ \\ \hline
 $U_1$ &       &  r,w  &  (r)  &  (r)  &   -   &       &       \\ \hline
 $U_2$ &       &  r,w  &  (r)  &  r,w  &       &   -   &       \\ \hline
 $U_3$ &  r,w  &  r,w  &  r,w  &  (r)  &       &       &   -   \\ \hline
\end{tabular}
\end{center}
\end{table}

\begin{table}[h]
\caption{Final access control matrix}
\begin{center}
\begin{tabular}{l|c|c|c|c|c|c|c|}
       & $O_1$ & $O_2$ & $O_3$ & $O_4$ & $U_1$ & $U_2$ & $U_3$ \\ \hline
 $U_1$ &       &  r,w  &       &  (r)  &   -   &       &       \\ \hline
 $U_2$ &       &  r,w  &       &  r,w  &       &   -   &       \\ \hline
 $U_3$ &  r,w  &  r,w  &  r,w  &  (r)  &       &       &   -   \\ \hline
\end{tabular}
\end{center}
\end{table}

\begin{table}[h!]
\caption{Final security levels}
\begin{center}
\begin{tabular}{|c|c|c|c|}
$O_1$ & $O_2$ & $O_3$ & $O_4$ \\ \hline
  2   &   1   &   2   &   1   \\ \hline
\end{tabular}
\end{center}
\end{table}

\newpage
\subsection*{Questions}
\textbf{Can $U_3$ write $O_1$ after step one?} yes \\
\textbf{Can $U_2$ read $O_1$ after step five?} no \\
\textbf{Can $U_1$ access the content of $O_3$ after step five?}
not directly, it is copied into $O_4$ \\
\textbf{Can $U_3$ chaneg the permissions on $O_4$ after step four?}
yes \\
\textbf{Additional policy:} The security level of an object must not be
decreased. \\
\textbf{sufficient for a secure system?} \\
Nein, da Inhalte höherer
Sicherheitsstufe(n) problemlos (z.~B. durch fehlende Schreibregelung) von High
nach Low gelangen.

\section{Rainbow tables}

\paragraph{Ein gültiges Passwort ist}
sfhuvigt

\paragraph{gefunden in der Kette}
(iamyourg -$>$ cgie)

\paragraph{in folgenden Schritte}
\begin{itemize}
\item 5 Iterationen $h(r(hash))$ des Eingabe-Hashes
\item Testen der Kette (shouldud -$>$ ycea), Länge 8
\item Testen der Kette (password -$>$ mqso), Länge 8
\item Testen der Kette (geheimni -$>$ cgie), Länge 8
\item Gefunden in (iamyourg -$>$ cgie) nach 3 Iterationon
\end{itemize}

\paragraph{Problem} ist die Qualität der gewählten Reduktionsfunktion. Sie führt
bereits nach wenigen Iterationen zu wiederkehrenden Hashes.


\section{Basic assembler (optional)}

\subsection*{Convert $0xBEEF$}
\begin{eqnarray}
    0xBEEF &=>& 0xB\cdot16^3 + 0x0E\cdot16^2 + 0x0E\cdot16^1 + 0x0F\cdot16^0 \\
           &=& 11\cdot16^3 + 14\cdot16^2 + 0x0E\cdot16^1 + 15 \\
           &=& 45056 + 3584 + 224 + 15 \\
           &=& 48879
\end{eqnarray}
im Binären reicht die einfache Konkatenation der einzelnen Binärwerte
\begin{eqnarray}
    0xBEEF &=>& 0x0B || 0x0E || 0x0E || 0x0F \\
           &=>& 1011 || 1110 || 1110 || 1111 \\
\end{eqnarray}

\subsection{Pointer}
\begin{itemize}
    \item IP ist der Instruction Pointer und zeigt auf den nächsten Befehl im
        Speicher, der ausgeführt werden soll.
    \item SP ist der Stack Pointer und zeigt auf den Beginn des Stacks innerhalb
        einer Funktion
    \item BP ist der Base Pointer zeigt auf den ursprünglichen Stack vor Aufruf.
        So können innerhalb einer Funktion Parameter referenziert werden.
\end{itemize}
Das vorangestellte E bzw. R kennzeichnen Pointer für 32-bit und analog
64-bit Architekturen. Folglich handelt es sich bei dem Code in
Listing~\ref{lst:hello.asm} um eine 64bit-Anwendung.

\lstinputlisting[caption={Assambler Snippet},label={lst:hello.asm}]{./e01/hello.asm}

\end{document}
