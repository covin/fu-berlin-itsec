\include{tex/00-header-ex}
\usepackage{ucs}
\usepackage[utf8x]{inputenc}
\usepackage{ngerman}
\usepackage[ngerman]{babel}
\usepackage[ngerman]{translator}
\usepackage{color}
\usepackage{url}

\usepackage{graphicx}
\usepackage{algorithmic}
\usepackage{hyperref}

\usepackage{textcomp}

\usepackage{makeidx}
\usepackage{amsmath}
\usepackage{amsfonts}
\usepackage{amssymb,euscript}
\usepackage{dsfont}
\usepackage{fancyhdr}
\usepackage{listings}
\usepackage{enumerate}

\pagestyle{empty}

\lstset{numbers=left,
        basicstyle=\footnotesize,
        numberstyle=\footnotesize,
        stepnumber=5,
        numbersep=5pt,
        frame=tb,
        framexleftmargin=5mm,
        xleftmargin=5mm
}

\pagestyle{fancy}
\lhead{Rechnersicherheit}
\chead{Übung \nr\\\today}
\rhead{Alexander Münn}


\newcommand{\nr}{05}

\begin{document}

\section{Informational flow and entropie}
\subsection{Compute entropy $H(X)$}
\begin{eqnarray}
    H(X) &=& \sum_{p \in X} p(x)\log{\frac{0}{p(x)}}  \nonumber \\
         &=& \frac{1}{2}\log{2} + 2 \cdot (\frac{1}{4}\log{4}) \nonumber \\
         &=& \frac{3}{2}
\end{eqnarray}

\subsection{Compute entropy $H(X|Y')$}
\paragraph{Für k=0} werden zunächst Wahrscheinlichkeiten $P[Y]$ sowie $P[X|Y]$
benötigt.
\begin{eqnarray}
    P[Y=1] &=& \frac{1}{2} \\
    P[Y=0] &=& \frac{1}{2} \\
    P[X=0|Y=1] &=& 0 \\
    P[X=1|Y=1] &=& P[X=2|Y=1] = \frac{1}{2} \\
    P[X=0|Y=0] &=& 1 \\
    P[X=1|Y=0] &=& P[X=2|Y=0] = 0 \\
\end{eqnarray}

Anhand der Wahrscheinlichkeiten lässt sich die ungewissheit über $X$ bei einem
bekannten $Y$ bestimmen:
\begin{eqnarray}
    H(X|Y) &=& \sum_{y \in Y} p(y) H(X|Y=y) \nonumber \\
           &=& \frac{1}{2} H(X|Y=0) + \frac{1}{2} H(X|Y=1) \\
    H(X|Y=0) &=& \sum_{p \in P[X|Y=0]}p(x)\log(\frac{1}{p(x)}) \\
             &=& 0 + 1\log{1} = 0 \\
    H(X|Y=1) &=& 0 + 2 \cdot \left( \frac{1}{2} log(2)\right) = 1 \\
    H(X|Y) &=& \frac{1}{2}\cdot 0 + \frac{1}{2}\cdot 1 = 0.5
\end{eqnarray}

\paragraph{Aus k=1} folgt eine andere Wahrscheinlichkeit für das auftreten von
$P[Y=0]$ bzw. $P[Y=1]$. Somit ändern sich auch die bedingten
Wahrscheinlichkeiten.
\begin{eqnarray}
    P[Y=1] &=& \frac{1}{4} \\
    P[Y=0] &=& \frac{3}{4} \\
    P[X=0|Y=1] &=& P[X=1|Y=1] = 0 \\
    P[X=2|Y=1] &=& 1 \\
    P[X=0|Y=0] &=& \frac{2}{3} \\
    P[X=1|Y=0] &=& \frac{1}{3} \\
    P[X=2|Y=0] &=& 0 \\
\end{eqnarray}
Die Ungewissheit über $X$ lässt sich mit den aktuellen werden wie unter $k=0$
bestimmen.
\begin{eqnarray}
    H(X|Y) &=& \sum_{y \in Y} p(y) H(X|Y=y) \nonumber \\
           &=& \frac{3}{4} H(X|Y=0) + \frac{1}{4} H(X|Y=1) \\
    H(X|Y=0) &=& \sum_{p \in P[X|Y=0]}p(x)\log \left( \frac{1}{p(x)} \right) \nonumber \\
           &=& \frac{2}{3}\log\frac{3}{2} + \frac{1}{3} \log{3} \\
    H(X|Y=1) &=&  1 \log(1) = 0 \\
    \label{eq:foo}
    H(X|Y) &=& \frac{3}{4}\cdot
                  \left(\frac{2}{3}\log\frac{3}{2} + \frac{1}{3} \log{3} \right)
               + \frac{1}{4} \cdot 0 \nonumber \\
           &\approx& 0.68872
\end{eqnarray}

\section{Informational flow and entropie}
Aus den kombinierten Wahrscheinlichkeiten von $P[X]$ und $P[Y]$ und den
Kombinationsmöglichkeiten \\
\begin{center}
\begin{tabular}{|cc|c|}
       \hline
       $X$ & $Y$ & $Z$ \\ \hline
        0  &  0  &  0  \\ \hline
        1  &  0  &  1  \\ \hline
        0  &  1  &  1  \\ \hline
        1  &  1  &  1  \\ \hline
\end{tabular}
\end{center}

ergeben sich:
\begin{eqnarray}
    P[Z=0] &=& \frac{1}{4} \\
    P[Z=1] &=& \frac{3}{4} \\
    P[X=1|Z=1] &=& P[Y=1|Z=1] = \frac{2}{3} \\
    P[X=0|Z=1] &=& P[Y=0|Z=1] = \frac{1}{3} \\
    P[X=0|Z=0] &=& 0 \\
    P[X=1|Z=0] &=& 0
\end{eqnarray}

Die Werte sind identisch zur Rechnung~\ref{eq:foo}. Auf die Abbildung möchte ich
an dieser Stelle verzichten. Da $X$ und $Y$ von einander unabhängig auftreten und
$P[X] = P[Y]$ ist, sind $H(X|Z') = H(Y|Z') \approx 0.68872$


\section{Informational flow - Pseudo Code}
\label{sec:aufgabe3}


\begin{algorithm}
\caption{Variante 1 -- Bits per Modulo Operation auslesen}
\begin{algorithmic}
    \State $x \gets [0,2^{64}-1]$
    \State $y \gets 0 $
    \State $z \gets 0 $
    \While {$x \neq 0 $}
        \If {$x \mod 2$}
            \State $z \gets 1$
        \EndIf
        \State shift $x$ right
        \Comment indirect violation here!?

        \If {$z$}
            \State $y += 1$
        \EndIf
        \State shift $y$ left
        \State $z \gets 0$
    \EndWhile
    \Comment $y$ contains bits of $x$ in reverse order
\end{algorithmic}
\end{algorithm}

\begin{algorithm}
\caption{Variante 2 -- bitweise logisch auslesen}
\begin{algorithmic}
    \State $x \gets [0,2^{64}-1]$
    \State $y \gets 0 $
    \State $z \gets 1 $
    \For {$i = 0 \to 63$}
        \If {$x \lor z$}
            \State $y \gets (y \lor z)$
        \EndIf
        \State shift $z$ left
    \EndFor
\end{algorithmic}
\end{algorithm}

\begin{algorithm}
\caption{Variante 3 -- numerisch logisch auslesen}
\label{psd:v3}
\begin{algorithmic}
    \State $x \gets [0,2^{64}-1]$
    \State $y \gets 0 $
    \State $z \gets 2^{63} $
    \While {$z \neq 0$}
        \If {$x \geq (y \lor z)$}
            \State $y \gets (y \lor z)$
        \EndIf
        \State shift $z$ right
    \EndWhile
\end{algorithmic}
\end{algorithm}


\section{Assembly indirect information flow}
Listing~\ref{lst:asm} ist eine Implementierung des Pseudocodes~\ref{psd:v3} aus
Aufgabe~\ref{sec:aufgabe3}
\lstinputlisting[caption={TheCode},label={lst:asm}]{e05/iflow.s}

\section{Upper and lower bounds}

\begin{equation}
\begin{aligned}
A \oplus B &= H          & A \otimes B &= Low \\
B \oplus I &= I          & B \otimes I &= B \\
B \oplus C &= High       & B \otimes C &= Low \\
A \oplus C \oplus D &= G & A \otimes C \otimes D &= A \\
A \oplus B \oplus D &= H & A \otimes B \otimes D &= Low
\end{aligned}
\end{equation}

\end{document}
